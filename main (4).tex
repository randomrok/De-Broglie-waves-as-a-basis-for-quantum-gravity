\documentclass{article}
\usepackage{graphicx} % Required for inserting images
\usepackage{amsmath}
\usepackage{amssymb}

\title{De-Broglie matter waves as the basis for quantum gravity}
\author{Elijah Gardi}
\date{July 2024}

\begin{document}

\maketitle

\section{Introduction}
There are a number of interpretations of quantum mechanics that fail to maintain realism, nor explain quantum non-locality. The proposed interpretation aims to maintain realism, whilst explaining quantum non-locality.\\
The extension of brownian motion to the quantum scale can only be achieved with matter wave perturbations combining to create a wave group (particle) contributed by the vacuum travelling at c. This means that the group velocity that we measure as the velocity of the particle, is the result of projecting the 3 dimensional velocity vector onto a one dimensional measurement axis. This results in a velocity vector less than c as described mathematically below.\\
Fundamentally, this interpretation proposes that the direction of a mass's velocity is stored as a complex number in the regular Schrodinger equation. This implies that the degrees of freedom in the system are equal to the dimension of the algebra used to store the direction vector. It also proposes that a quaternion algebra can be used to represent the direction vectors more generally, for 3D vacuum wave groups and that matter waves are the pilot wave/vacuum pressure. The complex number space of the Schrodinger equation can be used to construct the Hamilton space representation of the 3D velocity vector. This solution occurs for a perfect fluid stress-energy tensor where matter waves travel at $v=c$ (see $E=mc^2$ and $E=hc/\lambda) \implies m=h/\lambda c$).\\
\\
If there were a wave function whose solutions were made of components contributed from the other dimensions, it might best be represented as a combination of complex numbers or of the basis vectors from the quaternion algebra.

$$\Psi=f(x,y,z,t)=A e^{i(k_x x-w_x t)} e^{j(k_y y-w_y t)} e^{k(k_z z-w_z t)}$$
\\
This concept is explored more thoroughly below. 
\section{Maths}
\subsection{Two dimensional wave function superposition}

The formula for a normal wave function can be generated by neglecting two of the vacuum contributions to the quaternion wave function.

$$\Psi=f(x,t)=A e^{i(k_x x-w_x t)}$$

This is a plane wave solution for the matter wave along the x-axis. It is not a general solution as the other two dimensions contribute across an infinite range of values to the system. For one, this produces wave packets that are virtually instantaneously made of superpositions/interference of matter waves from the other two spatial dimensions. Over any length of time greater than the matter-wave length divided by c: 

$$t>t_0=\lambda /c=h/mc^2$$

The superposition of the other two dimensions' matter waves occurs (from vacuum pertubations). This is the time it takes light to move the distance $\lambda$ and is understood to be the matter wave of the vacuum pertubation (along the x-axis in this case) that travels at c.  $t_0=h/m_e c \approx 8*10^{-21}s$ for the electron. Maintaining the single axis measure of the angle, the resulting wave function can be written as:

$$\Psi=f(x,t)=A_0 e^{i(k_0 x-w_0 t)}+A_1e^{i(k_1 x-w_1 t)}$$

There are two ways of viewing the complex vector space that makes up a wave group in the original Schrodinger equation. We can imagine the complex number as representing the angle of the instantaneous 4D Hamilton velocity vector projected onto a 2D complex velocity vector (which is harder to think about) or we can think of the two dimensions it spans as the number of orthogonal matter waves contributing to the 3D velocity vector of the wave group. Both are equivalent as will be shown.\\ 
\\
The normal Schrodinger equation belongs to the complex number space. This also means it covers a two dimensional vector space over the real numbers. Secondly, the second wavefunction also belongs to a complex number space, but not the same one. When superimposed, they  allow the construction of a two dimensional vector space over the complex numbers $\mathbb{C}^2$ or the Hamilton space $\mathbb{H}$. This is like thinking how the orthogonal matter waves to both 2D wave groups velocity vectors lie in different (rotated) reference axes and their combination leads to a four dimensional vector space which can now store the resulting positive and negative values of the velocity vector's direction. Let the superimposed vacuum matter wave functions' velocity vectors belong to a two dimensional vector space over the complex numbers.
$$((a+bi),(c+di))$$
and choose a different set of complex basis vectors 1 and j, then:
$$(a+bi)1+(c+di)j = a+bi+cj+dij$$
Which, after defining $j^2=-1$ and $ij=-ji=k$ results in the same multiplication rules as the quaternions.\\

For $a=A_0$, $b=(k_0 x-w_0t)$, $c=A_1$ and $d=(k_1 x-w_1t) \implies$

$$v=A_0+(k_0 x-w_0t)i+A_1j+(k_1 x-w_1t)k$$

Since $A_1$ is an arbitrary constant we can choose this equal to $(k_3 x-w_3t)$ giving 


Using the quaternion representation we can project the other two components onto the complex space $\mathbb{C}$ 



The dot product of velocity (direction) vectors:

$$v_1 \circ v_2=c(a_1, b_1)\circ c(a_2, b_2)=c(a_1 a_2 + b_1 b_2)<c$$

Which is the same as:

$$\lvert v^2 \rvert =v^*v=c(a_1+ib_1)c(a_2-ib_2)=c(a_1 a_2+b_1 b_2)<c$$

This is zero when the vectors are orthogonal. This means the velocity measurement direction is only along a single axis. The maximum occurs when they are colinear giving a value of $v=c$. This will never happen since the vacuum contributions are orthogonal to the velocity direction vector and both must contribute for a group velocity to be defined. A good way to think about the vacuum contributions with respect to the velocity vector is to imagine the complex number representing the direction of the velocity vector, then comparing it to the additional terms in the quaternion representation.\\

Here we compute the projection of the quaternion basis elements onto the complex space, by calculating the dot product.

$$v^*v=v_\mathbb{C} \circ v_\mathbb{H} = $$



$$A_1=(k_2 )$$

$$\Psi=$$

This velocity definition makes it possible to define a group and phase velocity for the superimposed matter waves. Let N be the ratio of the time the two matter waves superimpose on the original wave:
$$N=t/t_0=\frac{\lambda_1/c+\lambda_2/c}{\lambda_0/c}=\frac{\lambda_1+\lambda_2}{\lambda_0}=\frac{h/mv+h/mv}{h/mc}=v/c^2$$

$$v_p=$$

For values of time greater than $t_0$:

$$t>t_0=\lambda /c=h/mc^2$$

The number of superpositions becomes too great for differentiation and we get a wave packet, which satisfies the uncertainty principle.

$$\Psi=f(x,t)=\sum_{n=1}^N{A_n e^{i(k_n x-w_n t)}}$$ 
where $m=t/t_0=\lambda /c $

For values much much greater than $t_0$:

$$t>>t_0=\lambda /c=h/mc^2$$


A more notationally satisfying method may be to construct the quaternions from pairs of complex numbers ($\mathbb{C}^2$) as in the Cayley-Dickson construction. The basis vectors 1 and j are multiplied by the 

\subsection{Phase and group velocities}

$$\Psi=f(x,t)=A_2 e^{i(k_1 x-w_1 t)}e^{i(k_2 x-w_2 t)}$$

$$\Psi=f(x,t)=\prod_{n=1}^N{A_n e^{i(k_n x-w_n t)}}=$$

\subsection{Direction of matter wave (quaternion)}
The direction of the matter wave can be computed by filling a quaternion with the dimensionless elements of the wavefunction, namely:

$$q=(A, k_x x-w_x t, k_y y-w_y t, k_z z-w_z t)$$
\subsection{Stress energy tensor resulting}
Computing the stress energy tensor
Representing the quaternion as a column vector allows one to compute the conjugate as a 4x4 matrix.

$$q=
\left[
\begin{matrix}
A&k_x x-w_x t&k_y y-w_y t&k_z z-w_z t
\end{matrix}
\right]
$$

$$q^*=
\left[
\begin{matrix}
A\\k_x x-w_x t\\k_y y-w_y t\\k_z z-w_z t
\end{matrix}
\right]
$$

Then the stress energy tensor becomes:
$$T=qq^*$$
\section{Conclusion}
Implications for cosmology

\end{document}
