\documentclass{article}
\usepackage{graphicx} % Required for inserting images
\usepackage{amsmath}

\title{De-Broglie matter waves as a basis for quantum gravity}
\author{Elijah Gardi}
\date{July 2024}

\begin{document}

\maketitle

\section{Introduction}
There are a number of interpretations for the 

\section{Maths}
\subsection{Four dimensional wave function}

If there were a wave function made up of components contributed from the other dimensions, it might best be represented as a combination of complex numbers or of the basis vectors from the quaternion algebra.

$$\Psi=f(x,y,z,t)=A e^{i(k_x x-w_x t)} e^{j(k_y y-w_y t)} e^{k(k_z z-w_z t)}$$

The normal formula for a normal wave function can be generated by neglecting two of the vacuum contributions to the wave function.

$$\Psi=f(x,t)=A e^{i(k_x x-w_x t)}$$
\subsection{Direction of matter wave (quaternion)}
The direction of the matter wave can be computed by filling a quaternion with the dimensionless elements of the wavefunction, namely:

$$q=(A, k_x x-w_x t, k_y y-w_y t, k_z z-w_z t)$$
\subsection{Stress energy tensor resulting}
Computing the stress energy tensor
Representing the quaternion as a column vector allows one to compute the conjugate as a 4x4 matrix.

$$q=
\left[
\begin{matrix}
A&k_x x-w_x t&k_y y-w_y t&k_z z-w_z t
\end{matrix}
\right]
$$

$$q^*=
\left[
\begin{matrix}
A\\k_x x-w_x t\\k_y y-w_y t\\k_z z-w_z t
\end{matrix}
\right]
$$

Then the stress energy tensor becomes:
$$T=qq^*$$
\section{Conclusion}
Implications for cosmology

\end{document}
