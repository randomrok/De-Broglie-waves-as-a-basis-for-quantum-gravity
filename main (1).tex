\documentclass{article}
\usepackage{graphicx} % Required for inserting images
\usepackage{amsmath}
\usepackage{amssymb}

\title{De-Broglie matter waves as the basis for quantum gravity}
\author{Elijah Gardi}
\date{July 2024}

\begin{document}

\maketitle

\section{Introduction}
There are a number of interpretations of quantum mechanics that fail to maintain realism, nor explain quantum non-locality. The proposed interpretation aims to maintain realism, whilst explaining quantum non-locality. Fundamentally, this method proposes that the direction of a mass is stored as a complex number in the regular Schrodinger equation. This implies that the degrees of freedom in the system are equal to the dimension of the algebra used to store the direction vector. It also proposes that a quaternion algebra can be used to represent the direction vectors more generally and that matter waves are the pilot wave/vacuum pressure. This solution occurs for a perfect fluid stress-energy tensor where matter waves travel at $v=c$ (see $E=mc^2$ and $E=hc/\lambda) -> m=h/\lambda c$).

\section{Maths}
\subsection{Four dimensional wave function}

If there were a wave function whose solutions were made of components contributed from the other dimensions, it might best be represented as a combination of complex numbers or of the basis vectors from the quaternion algebra.

$$\Psi=f(x,y,z,t)=A e^{i(k_x x-w_x t)} e^{j(k_y y-w_y t)} e^{k(k_z z-w_z t)}$$

The formula for a normal wave function can be generated by neglecting two of the vacuum contributions to the wave function.

$$\Psi=f(x,t)=A e^{i(k_x x-w_x t)}$$

This solution is not general as the other two dimensions contribute across an infinite range of values to the system. For one, this produces wave packets that are instantaneously made of super-positions/interference of matter waves from the other two spatial dimensions. Over periods of time greater than the matter-wave length of the contributing dimension divided by c: 
$$t>t_0=\lambda /c=h/mc^2$$
The superposition of another matter wave occurs. 
$$\Psi=f(x,t)=A_2 e^{i(k_1 x-w_1 t)}e^{i(k_2 x-w_2 t)}$$
For values of time much much greater than $t_0$:
$$t>>t_0=\lambda /c=h/mc^2$$
The number of superpositions becomes too great for differentiation and we get a wave packet.
$$\Psi=f(x,t)=A_2 e^{i(k_1 x-w_1 t)}e^{i(k_2 x-w_2 t)}$$

A more notationally satisfying method may be to construct the quaternions from pairs of complex numbers ($\mathbb{C}^2$) as in the Cayley-Dickson construction. The basis vectors 1 and j are multiplied by the 
\subsection{Direction of matter wave (quaternion)}
The direction of the matter wave can be computed by filling a quaternion with the dimensionless elements of the wavefunction, namely:

$$q=(A, k_x x-w_x t, k_y y-w_y t, k_z z-w_z t)$$
\subsection{Stress energy tensor resulting}
Computing the stress energy tensor
Representing the quaternion as a column vector allows one to compute the conjugate as a 4x4 matrix.

$$q=
\left[
\begin{matrix}
A&k_x x-w_x t&k_y y-w_y t&k_z z-w_z t
\end{matrix}
\right]
$$

$$q^*=
\left[
\begin{matrix}
A\\k_x x-w_x t\\k_y y-w_y t\\k_z z-w_z t
\end{matrix}
\right]
$$

Then the stress energy tensor becomes:
$$T=qq^*$$
\section{Conclusion}
Implications for cosmology

\end{document}
